\documentclass[]{article}
\usepackage{lmodern}
\usepackage{amssymb,amsmath}
\usepackage{ifxetex,ifluatex}
\usepackage{fixltx2e} % provides \textsubscript
\ifnum 0\ifxetex 1\fi\ifluatex 1\fi=0 % if pdftex
  \usepackage[T1]{fontenc}
  \usepackage[utf8]{inputenc}
\else % if luatex or xelatex
  \ifxetex
    \usepackage{mathspec}
  \else
    \usepackage{fontspec}
  \fi
  \defaultfontfeatures{Ligatures=TeX,Scale=MatchLowercase}
  \newcommand{\euro}{€}
\fi
% use upquote if available, for straight quotes in verbatim environments
\IfFileExists{upquote.sty}{\usepackage{upquote}}{}
% use microtype if available
\IfFileExists{microtype.sty}{%
\usepackage{microtype}
\UseMicrotypeSet[protrusion]{basicmath} % disable protrusion for tt fonts
}{}
\usepackage[margin=1in]{geometry}
\usepackage{hyperref}
\PassOptionsToPackage{usenames,dvipsnames}{color} % color is loaded by hyperref
\hypersetup{unicode=true,
            pdftitle={Matrix Algebra - Refresher},
            pdfauthor={T. Markaryan},
            pdfborder={0 0 0},
            breaklinks=true}
\urlstyle{same}  % don't use monospace font for urls
\usepackage{graphicx,grffile}
\makeatletter
\def\maxwidth{\ifdim\Gin@nat@width>\linewidth\linewidth\else\Gin@nat@width\fi}
\def\maxheight{\ifdim\Gin@nat@height>\textheight\textheight\else\Gin@nat@height\fi}
\makeatother
% Scale images if necessary, so that they will not overflow the page
% margins by default, and it is still possible to overwrite the defaults
% using explicit options in \includegraphics[width, height, ...]{}
\setkeys{Gin}{width=\maxwidth,height=\maxheight,keepaspectratio}
\setlength{\parindent}{0pt}
\setlength{\parskip}{6pt plus 2pt minus 1pt}
\setlength{\emergencystretch}{3em}  % prevent overfull lines
\providecommand{\tightlist}{%
  \setlength{\itemsep}{0pt}\setlength{\parskip}{0pt}}
\setcounter{secnumdepth}{0}

%%% Use protect on footnotes to avoid problems with footnotes in titles
\let\rmarkdownfootnote\footnote%
\def\footnote{\protect\rmarkdownfootnote}

%%% Change title format to be more compact
\usepackage{titling}

% Create subtitle command for use in maketitle
\newcommand{\subtitle}[1]{
  \posttitle{
    \begin{center}\large#1\end{center}
    }
}

\setlength{\droptitle}{-2em}
  \title{Matrix Algebra - Refresher}
  \pretitle{\vspace{\droptitle}\centering\huge}
  \posttitle{\par}
  \author{T. Markaryan}
  \preauthor{\centering\large\emph}
  \postauthor{\par}
  \predate{\centering\large\emph}
  \postdate{\par}
  \date{May 11, 2016}


\usepackage{bbm}
\usepackage{threeparttable}
\usepackage{booktabs}
\usepackage{lipsum}
\usepackage{tcolorbox}
\usepackage{tkz-graph}
\newcommand{\vect}[1]{\boldsymbol{#1}}

\newtcolorbox{pression}[2][]{enhanced,title=My title,
attach boxed title to top center={yshift=-3mm,yshifttext=-1mm},attach boxed title to top left={xshift=1cm,yshift=-2mm},
boxed title style={size=small,colupper=black},
title={#2},#1}

\tcbuselibrary{skins}
\usetikzlibrary{shadings}
\tcbset{
    skin=enhanced,
    fonttitle=\bfseries,
    interior style={white},
    segmentation style={black,solid,opacity=0.2,line width=1pt}
}

\GraphInit[vstyle = Shade]
\tikzset{
  LabelStyle/.style = { rectangle, rounded corners, draw,
                        minimum width = 2em, fill = green!30,
                        text = red, font = \bfseries },
  VertexStyle/.append style = { inner sep=3pt,
                                font = \Large\bfseries},
  EdgeStyle/.append style = {->, bend left} }
\thispagestyle{empty}

% Redefines (sub)paragraphs to behave more like sections
\ifx\paragraph\undefined\else
\let\oldparagraph\paragraph
\renewcommand{\paragraph}[1]{\oldparagraph{#1}\mbox{}}
\fi
\ifx\subparagraph\undefined\else
\let\oldsubparagraph\subparagraph
\renewcommand{\subparagraph}[1]{\oldsubparagraph{#1}\mbox{}}
\fi

\begin{document}
\maketitle

A special value of matrices and Matrix Algebra is that they nable to
express many operations arising in mathematics, statistics and many
other fields in a clear concise manner. Matrices also enhance the
understanding of which apsects generalize to higher dimensions.

\section{1. Matrices as
Transformations}\label{matrices-as-transformations}

\paragraph{\texorpdfstring{\(R \to R\)}{R \textbackslash{}to R}}\label{r-to-r}

\[y=ax\]

A single number \(\underline{a}\) represents linear transformation.

\paragraph{\texorpdfstring{\(R^2 \to R\)}{R\^{}2 \textbackslash{}to R}}\label{r2-to-r}

\[y=a_1x_1+a_2x_2\]

Vector \(\mathbf{A}=(a_1, a_2)\) represents linear transformation.

\paragraph{\texorpdfstring{\(R^2 \to R^2\)}{R\^{}2 \textbackslash{}to R\^{}2}}\label{r2-to-r2}

\[\begin{cases} y_1 = a_{11}x_1 + a_{12}x_2\\ y_2 = a_{21}x_1 + a_{22}x_2 \end{cases}\]

Matrix
\(\begin{pmatrix} a_{11} & a_{12} \\ a_{21} & a_{22} \end{pmatrix}\)
represents linear transformation.

\begin{tcolorbox}[colback=yellow!5,colframe=yellow!40!black,title=Questions]
1. What matrix corresponds to a linear mapping?

2. What matrix corresponds to a linear maping $R^{10} \to R^{20}?$
\end{tcolorbox}

Thus, for linear mappings:

\begin{itemize}
\item
  \(R \to R\) \textbar{} given by numbers
\item
  \(R^{n} \to R^{m}\) \textbar{} given by matrices
\end{itemize}

So, we can write compactly

\begin{equation}\label{eqn:matrix}\underline{y}=A\underline{x}\end{equation}

where

\[\underline{y}=\binom{y_1}{y_2},  A = \begin{pmatrix} a_{11} & a_{12} \\ a_{21} & a_{22} \end{pmatrix} , \underline{x}=\binom{x_1}{x_2}\]

Note that equation \eqref{eqn:matrix} looks similar to the linear
equation in the case of \(R \to R\)

In general, for transformation \(R^{n} \to R^{m}\), a matrix
\[\begin{pmatrix}A\end{pmatrix}_{m*n}\] reprezents a linear
transformation.

Viewing matrices as linear transformations can help understand,
seemingly strange definition of a product of matrices as well as:

\begin{itemize}
\item
  Addition of matrices
\item
  Multiplication by a scalar
\item
  Inverse
\end{itemize}

\section{\texorpdfstring{2. Notation and Definitions\\
}{2. Notation and Definitions }}\label{notation-and-definitions}

\begin{tcolorbox}[colback=green!5,colframe=red!40!black,title=Definition]
A matrix is an $m*n$ array of numbers with  $\underline{m}$ rows and  $\underline{n}$ columns.
\end{tcolorbox}

~

Matrices are usually denoted by capital's: A,B, C\ldots{}

Generic elemment of matrix is written as:

\begin{center}
$\begin{pmatrix}A\end{pmatrix}_{ij}$ or $a_{ij}$, where  $i=1,...,m ; j=1,...n$
\end{center}

Examples of Matrices:

\begin{itemize}
\item
  \(\mathbf{1x1}\) marix is a \(\underline{scalar}\)
\item
  \(\mathbf{nx1}\) marix is a \(\underline{column vector}\)
\item
  \(\mathbf{1xn}\) marix is a \(\underline{row vector}\)
\item
  \(\mathbf{nxn}\) marix is a \(\underline{square matrix}\)
\end{itemize}

We denote column vectors by \(a, b, z...\) and to emphasize that they
are vectors we use:\\
 \(\underline{a}, \underline{b}, \underline{z}\).\\

\(\mathbf{Equality}\) of matrices: \(A=B\)

\begin{itemize}
\tightlist
\item
  They are the same size
\item
  \(A_{ij}=B_{ij}\)
\end{itemize}

\(\mathbf{Transpose}\) of matrix:

\begin{itemize}
\tightlist
\item
  If \(A\) is a \(m*n\) matrix, then \((A')_{ij}\) = \((A)_{ij}\) is a
  \(n\)x\(m\) matrix.\\
\end{itemize}

Examples of transpose will be provide.\\

\(\mathbf{Symmetric}\) \(A'=B\)\\

\section{3. Matrices as
Transformations}\label{matrices-as-transformations-1}

\paragraph{\texorpdfstring{Addition of Matrices\\
}{Addition of Matrices }}\label{addition-of-matrices}

\begin{tcolorbox}[colback=green!5,colframe=red!40!black,title=Definition]
If $A$ and $B$ are $mxn$ matrices then $C=A+B$ is defined as elementwise addition.

$(C)_{ij}$ = $(A)_{ij}$ + $(B)_{ij}$       $\forall$            ${i,j}$
\end{tcolorbox}

~

\(\mathbf{Examples}\)\\

\begin{enumerate}
\def\labelenumi{\alph{enumi})}
\item
  A= \(\begin{pmatrix} 1 & 2 \\ 3 & 4 \end{pmatrix}\) B=
  \(\begin{pmatrix} -1 & 1 \\ 1 & -4 \end{pmatrix}\);
  A+B=\(\begin{pmatrix} 1+(-1) & 2+1 \\ 3+1 & 4+(-4) \end{pmatrix}\)\\
\item
  A= \(\begin{pmatrix} 1 & 3 & 5 \\ 2 & 4 & 6\end{pmatrix}\) B=
  \(\begin{pmatrix} -1 & -3 & -5 \\ -2 & -4 & -6\end{pmatrix}\);
  A+B=\(\begin{pmatrix} 0 & 0 & 0 \\ 0 & 0 & 0\end{pmatrix}\)\\
\end{enumerate}

\(\textbf{Additive Identities }\)\\

Matrix \(O_{mxn}\) consisting of all \(O's\) is called
\(\textbf{Null Matrix}\) and serves as the additive identitiy.\\

\[A_{mxn}+O_{mxn}=A_{mxn}\]\\

-\textgreater{}
\(\text{\textit{Note: There is an additive identity for each size}}\)\\

\(\textbf{Properties of Matrix Addition}\)\\

\begin{itemize}
\tightlist
\item
  Commutative: \(A+B = B+A\)\\
\item
  Associative: \(A+(B+C) = (A+B)+C\) \(\text{\textit{Why?}}\)\\
\item
  Multiplication by a scallar: If \(A = A_{ixj}\) then
  \(B=cA=(c*A_{ixj})\)
\end{itemize}

\(\mathbf{Example}\)\\

4*\(\begin{pmatrix} 1 & 0 & -1 \\ 2 & 1 & 4\end{pmatrix}\) =
\(\begin{pmatrix} 4 & 0 & -4 \\ 8 & 4 & 16\end{pmatrix}\)\\

\section{\texorpdfstring{4. Multiplication of Matrices\\
}{4. Multiplication of Matrices }}\label{multiplication-of-matrices}

Matrix multiplication may first seem unnatural or strange, but viewing
it as a \(\textbf{composition}\) of two linear maps makes it completelly
clear.\\

\(\mathbf{Example}\)\\
 - One-dimensional case

\begin{center}
    \begin{tikzpicture}
      \SetGraphUnit{3}
      \Vertex[L=R]{B}
      \WE[L=R](B){A}
      \EA[L=R](B){C}
      \Edge[label = I](A)(B)
      \Edge[label = II](B)(C)
      \tikzset{EdgeStyle/.append style = {bend left = 50}}
      \Edge[label = III](A)(C)
    \end{tikzpicture}
  \end{center}

\(I\): \(y=\alpha*x\)\\[2\baselineskip] \(II\):
\(\zeta=\beta*y\)\\[2\baselineskip] \(III\):\(\zeta=\beta*\alpha*x\)

\begin{itemize}
\tightlist
\item
  \(\mathbbm{R^2}\) case
\end{itemize}

\[\mathbbm{R^2}\xrightarrow{--1--}\mathbbm{R^2}\xrightarrow{--2--}\mathbbm{R^2}\]
\[\overrightarrow{---3---}\]\\

1:\(\begin{cases} y_1 = x_1 + x_2\\ y_2 = x_1 - x_2 \end{cases}\)\\[2\baselineskip]
and,

\begin{equation}\label{eqn:matrix}\underline{y}=A\underline{x}\end{equation}

\[A = \begin{pmatrix} 1 & 1 \\ 1 & -1 \end{pmatrix}\]

2:
\(\begin{cases} \zeta_1= y_1 + 2y_2\\ \zeta_2 = y_1 - y_2 \end{cases}\)\\

and,

\begin{equation}\label{eqn:matrix}\underline{\zeta}=B\underline{y}\end{equation}

\[B = \begin{pmatrix} 1 & 2 \\ 1 & -1 \end{pmatrix}\]\\

3:\(\begin{cases} \zeta_1= x_1 + x_2 + 2x_1 - 2x_2 \\ \zeta_2 = x_1 + x_2 - x_1 + x_2 \end{cases}\)
\(\Longleftrightarrow\)
\(\begin{cases} \zeta_1= 3x_1 - x_2\\ \zeta_2 = 2x_2 \end{cases}\)\\

thus,

\begin{equation}\label{eqn:matrix}\underline{\zeta}=C\underline{x}\end{equation}

\[C = \begin{pmatrix} 3 & -1 \\ 0 & 2 \end{pmatrix}\]\\

So it is natural to define \(BA=C\), and we can easily verify that:

\[BA = \begin{pmatrix} 1 & 2 \\ 1 & -1 \end{pmatrix}\begin{pmatrix} 1 & 1\\ 1 & -1 \end{pmatrix}=\begin{pmatrix} 3 & -1 \\ 0 & 2 \end{pmatrix}\]\\

Identifying multiplication as a composition should make it natural that
matrix mulitplication is only defined for matrices:\\

\[\begin{pmatrix}A\end{pmatrix}_{m*n} \begin{pmatrix}B\end{pmatrix}_{n*p}\]

\[n=m\]\\
 -\textgreater{} Dimensions must conform before \(A*B\) can be
defined.\\

\[\mathbbm{R_p}\xrightarrow{--B--}\mathbbm{R_n}\xrightarrow{--A--}\mathbbm{R_m}\]
\[\overrightarrow{---AB---}\]\\

\begin{tcolorbox}[colback=green!5,colframe=red!40!black,title=Definition]
Let $\begin{pmatrix}A\end{pmatrix}_{m*n}$ and $\begin{pmatrix}B\end{pmatrix}_{n*p}$  then $C=AB$ is defined
as $\begin{pmatrix}C\end{pmatrix}_{i*j} = \sum\limits_{k=1}^n \mathcal A_{ik}B_{kj}$ ; i=1,...,m; j=1,...,p
\end{tcolorbox}

~

\(\mathbf{Examples}\)\\

\begin{enumerate}
\def\labelenumi{\alph{enumi})}
\tightlist
\item
  \(\begin{pmatrix} 3 & 1 & 10 \\ 2 & 2 & 8 \end{pmatrix}\)
  \(\begin{pmatrix} 60 \\ 80 \\ 40 \end{pmatrix}\) =
  \(\begin{pmatrix} 660 \\ 600 \end{pmatrix}\)\\
\end{enumerate}

-\textgreater{} The matrices are represented by the dimensions as
follows: 2x3, 3x1 and 2x1.\\

\begin{enumerate}
\def\labelenumi{\alph{enumi})}
\setcounter{enumi}{1}
\tightlist
\item
  \(\begin{pmatrix} 1 & 2 \\ 3 & 4 \end{pmatrix}\)
  \(\begin{pmatrix} 0 & -1 \\ 1 & -1 \end{pmatrix}\) =
  \(\begin{pmatrix} 2 & -3 \\ 4 & -7 \end{pmatrix}\)\\
\end{enumerate}

-\textgreater{} The matrices are represented by the dimensions as
follows: 2x2, 2x2 and 2x2.\\

\begin{enumerate}
\def\labelenumi{\alph{enumi})}
\setcounter{enumi}{2}
\item
  \(\begin{pmatrix} 0 & -1 \\ 1 & -1 \end{pmatrix}\)
  \(\begin{pmatrix} 1 & 2 \\ 3 & 4 \end{pmatrix}\) =
  \(\begin{pmatrix} -3 & -4 \\ -2 & -2 \end{pmatrix}\)\\
\item
  Inner Product\\
   \(\begin{pmatrix} 2 & 2 & 8 \end{pmatrix}\)
  \(\begin{pmatrix} 50 \\ 100 \\ 30 \end{pmatrix}\) = \(540\)\\
\end{enumerate}

-\textgreater{} The matrices are represented by the dimensions as
follows: 1x3, 3x1 and 1x1.\\

More generally, if
\(\begin{pmatrix} x_i \\ .\\ .\\.\\x_n \end{pmatrix}\)
\(\begin{pmatrix} y_i \\ .\\ .\\.\\y_n \end{pmatrix}\), then
\(X'Y=Y'X = \sum\limits_{i=1}^n \mathcal X_{i}Y_{i}\)

\begin{enumerate}
\def\labelenumi{\alph{enumi})}
\setcounter{enumi}{4}
\item
  Outer Product\\

  \(\begin{pmatrix} 50 \\ 100 \\ 30 \end{pmatrix}\)
  \(\begin{pmatrix} 2 & 2 & 8 \end{pmatrix}\) =
  \(\begin{pmatrix} 100 & 100 & 400 \\ 200 & 200 & 800 \\ 60 & 60 & 240 \end{pmatrix}\)\\
\end{enumerate}

-\textgreater{} The matrices are represented by the dimensions as
follows: 3x1, 1x3 and 3x3.\\

-\textgreater{}
\(\text{\textit{Note: b) $\neq$ c) which means that $\textbf{matrix mulitplication}$ is  $\textbf{NOT}$ commutative. Examples d) and e) confirm that.}}\)\\[2\baselineskip]

\begin{tcolorbox}[colback=green!5,colframe=red!40!black,title=Definition (Linear Forms)]
If if $\underline{x}=\begin{pmatrix} x_i \\ .\\ .\\.\\x_n  \end{pmatrix}$ is a vector of variables $x_1, ...x_n$ and $\underline{a}=if \begin{pmatrix} a_i \\ .\\ .\\.\\a_n  \end{pmatrix}$ is a vector of constants than $a'x=a_1x_1+a_2x_2+...a_nx_n$ is called a $\textbf{linear form}$ in x.



\end{tcolorbox}

It is easy to show that:

\[\frac{d}{dx}(a'x)=\underline{a}\]\\

\begin{tcolorbox}[colback=green!5,colframe=red!40!black,title=Definition (Length of a vector)]
If if $\underline{x}=\begin{pmatrix} x_i \\ .\\ .\\.\\x_n  \end{pmatrix}$ $\in \mathbbm{R^n}$ then, $\|x\|$=$\sqrt{x'x}$=$\sqrt{\sum\limits_{i=1}^n x^2_i}$, if $\|x\|$=1, then x has an unit length.




\end{tcolorbox}

\(\textbf{Quadratic Forms}\).

If A is symmetric then \(\underline{x'}A\underline{x}\) is called a
\(\textbf{quadratic form}\)\\

-\textgreater{} \(\text{\textit{Note: $x'Ax$ ia a scalar.}}\)\\

\(\mathbf{Examples}\)\\

\(\begin{pmatrix} x_1 & x_2 \end{pmatrix}\)
\(\begin{pmatrix} 1 & 3 \\ 3 & 2 \end{pmatrix}\)
\(\begin{pmatrix} x_1 \\ x_2 \end{pmatrix}\) =
\({x^{2}_1 + 6x_1x_2 + 2x^{2}_2}\)~
-\textgreater{}\(\text{\text{(quadratic form in 2 variables)}}\)\\

-In 1-dimensional case: \(x*a*x = ax^2\) \&
\(\frac{d}{dx}(x*a*x)=2ax\)\\

-In higher dimensions:
\(\frac{d}{dx}(\underline{x'}*A*\underline{x})=2A\underline{x}\)
-\textgreater{} \(\text{\textit{Show it!}}\)\\

\(\textbf{Definite and semi-definite matrices.}\)\\

\begin{tcolorbox}[colback=green!5,colframe=red!40!black,title=Definition]
A symmetric matrix $\underline{A}$ is said to be $\textbf{positive definite}$ if $\underline{x'}A\underline{x}$  $>$ for all $\underline{x}\neq0$



\end{tcolorbox}

-Positive semi-definite is defined by changing \(">"\) to \("\geq"\).

-Negative definite and negative semi-definite are defined using \("<"\)
and \("\leq"\), respectively.\\

\(\mathbf{Example}\)\\

-Variance-Covariance Matrices are Positive Semi-Definite.\\

\(\mathbf{Example}\)\\

\begin{enumerate}
\def\labelenumi{\alph{enumi})}
\tightlist
\item
  Show that:\\
\end{enumerate}

\(A=\begin{pmatrix} 2 & -2 \\ -2 & 2 \end{pmatrix}\) is positive
semi-definite.\\

\begin{enumerate}
\def\labelenumi{\alph{enumi})}
\setcounter{enumi}{1}
\tightlist
\item
  Show that:\\
\end{enumerate}

\(B=\begin{pmatrix} 4 & -2 & 0\\ -2 & 4 & -2 \\ 0 & -2 & 4\end{pmatrix}\)
is positive definite.\\

\begin{enumerate}
\def\labelenumi{\alph{enumi})}
\setcounter{enumi}{2}
\tightlist
\item
  Show that:\\
\end{enumerate}

\(C=\begin{pmatrix} 1 & 1 \\ 1 & -1 \end{pmatrix}\) is indefinite.\\

\(\textbf{Properties of Matrix Multiplication.}\)\\

-1. Associative:\\[2\baselineskip] \(A(BC)=(AB)C\)\\[2\baselineskip] -2.
Distributive:\\[2\baselineskip] \(A(B+C)=AB+AC\)\\[2\baselineskip]
\((B+C)A=BA+CA\)\\[2\baselineskip] -3. NOT-Commutative\\[2\baselineskip]
-4. Identity Matrix\\[2\baselineskip]
\(I_{nxn}\)=\(\begin{bmatrix} 1 & \cdots & 0 \\ \vdots & \ddots & \vdots \\ 0 & \cdots & 1 \end{bmatrix}\)
serves as multiplicative identity for square matrices for \(\forall\)
\(A_{nxn}\)

\[I_{nxn}A_{nxn}=A_{nxn}I_{nxn}=A_{nxn}\]\\

-\textgreater{}
\(\text{\textit{Note:In case of identity matrix product does not commute!}}\)\\

-v5. For any matrix \(A_{mxn}\)\\
 \[A_{mxn}O_{nxp}=O_{mxp}\]\\
 \[(AB)'=B'A'\] \[(ABC)'=C'B'A'\]\\

\section{\texorpdfstring{5. Inverse of Matrices\\
}{5. Inverse of Matrices }}\label{inverse-of-matrices}

\begin{tcolorbox}[colback=green!5,colframe=red!40!black,title=Definition]
  If for a squarmatrix $A_{nxn}$ $\exists$  a matrix $B_{nxn}$ sit.\
  
  (*) $AB = BA = I_{nxn}$.  Than $B$ is calles an $\underline{inverse}$ of $A$ and denoted $A^-1$. If $A$ has an inverse, it is called $\underline{non-singular}$.
  

\end{tcolorbox}

-It turns out that one needs to check only one of the 2 conditions in
(*).\\[2\baselineskip] -The inverse if it exists is
\(\underline{unique}\).\\

\(\mathbf{Example}\)\\
 a)\(A=\begin{pmatrix} 2 & -1 \\ -1 & 1 \end{pmatrix}\) then
\(A^{-1}=\begin{pmatrix} 1 & 1 \\ 1 & 2 \end{pmatrix}\) -\textgreater{}
check directly.\\

b)Show that \(A=\begin{pmatrix} 1 & 3 \\ 2 & 6 \end{pmatrix}\) does not
have an inverse.\\

c)Show that if \(A=\begin{pmatrix} a & b \\ c & d \end{pmatrix}\) than
\(A^{-1} =\frac{1}{ad-bc}\begin{pmatrix} d & -b \\ -c & a \end{pmatrix}\)
iff ad-bc \(\neq\) 0.\\

\paragraph{\texorpdfstring{Properties of Matrix Inverse\\
}{Properties of Matrix Inverse }}\label{properties-of-matrix-inverse}

\begin{enumerate}
\def\labelenumi{\alph{enumi})}
\tightlist
\item
  \((A')^{-1}=(A^{-1})'\)\\
\item
  \((A^{-1})^{-1}=A\)\\
\item
  \((AB)^{-1}=B^{-1}A^{-1}\)\\
\end{enumerate}

\paragraph{\texorpdfstring{Proof of c)\\
}{Proof of c) }}\label{proof-of-c}

Need to show that:\\
 \[(AB)(B^{-1}A^{-1})= I\]

Now:

\begin{align}
    (AB)(B^{-1}A^{-1}) &=A(BB^{-1})A^{-1}&& \text{(Associative}  \text{)}\\
     &=AIA^{-1} && BB^{-1}=I \\
     &=AA^{-1}   && BB^{-1}=I \\
     &=I &&\text{(Definition of Inverse}  \text{)}
\end{align}

\begin{enumerate}
\def\labelenumi{\alph{enumi})}
\setcounter{enumi}{2}
\tightlist
\item
  Also extends to more than 2 matrices. For example:\\
  \[(ABC)^{-1}=C^{-1}B^{-1}A^{-1}\]
\end{enumerate}

----\textgreater{} Show it!\\
 -Armed with definition of matrix inverse we can cast system of linear
equations and solution in the matrix form.\\

(*)
\[\begin{cases} x_1 + 2x_2 +x_3= 4\\ 4x_2 + 3x_3 = 3\\3x_1 + 6x_2 = -3 \end{cases}\]\\[2\baselineskip]
Denote:\\

\(A=\begin{pmatrix} 1 & 2 & 1\\ 0 & 4 & 3 \\ 3 & 6 & 0\end{pmatrix}\),
\(b=\begin{pmatrix} 4\\ 3 \\-3\end{pmatrix}\) ,
\(x=\begin{pmatrix} x_1\\ x_2 \\x_3\end{pmatrix}\)\\

We can write (*) as:\\[2\baselineskip] (**)
\[A\underline{x}=\underline{b}\]

Also, if (*) has an unique solution then:

\[\underline{x}=A^{-1}b\]

Which is how we solve in case of \(\mathbbm{R^1}\).\\

\begin{itemize}
\item
  We will return and solve this equation after introducing column
  reduction.
\item
  As we have already seen, not every square matrix has an inverse. A
  \textless{}=\textgreater{} condition for existence of the inverse is
  for the columns of the matrix to be
  \(\textbf{LINEARLY INDEPENDENT}\).\\
\end{itemize}

\paragraph{\texorpdfstring{Linear Independence\\
}{Linear Independence }}\label{linear-independence}

\begin{itemize}
\tightlist
\item
  Two vectors of same size:\\
\end{itemize}

\(a=\begin{pmatrix} a_1 \\ .\\ .\\.\\a_n \end{pmatrix}\) \&
\(b=\begin{pmatrix} b_1 \\ .\\ .\\.\\b_n \end{pmatrix}\)\\

Are called \(\textbf{LINEARLY INDEPENDENT}\) if for \(\forall\)
\(\lambda_1\), \(\lambda_2\), \(\in\) \(R\) siti
\(\lambda^{2}_1 + \lambda^{2}_2 > 0\)\\
 \[\lambda_1a + \lambda_2b \neq \underline{0}\],\\

Otherwise there are called \(\textbf{LINEARLY DEPENDENT}\).\\

\paragraph{\texorpdfstring{EXAMPLES\\
}{EXAMPLES }}\label{examples}

\begin{enumerate}
\def\labelenumi{\alph{enumi})}
\tightlist
\item
  \(a=\begin{pmatrix} 1\\ 2 \end{pmatrix}\) and
  \(b=\begin{pmatrix} 4\\ 8 \end{pmatrix}\) --\textgreater{} Ale
  Linearly Dependent\\
\end{enumerate}

choose \(\lambda_1 = -4\) \& \(\lambda_2 = 1\)\\

\begin{enumerate}
\def\labelenumi{\alph{enumi})}
\setcounter{enumi}{1}
\tightlist
\item
  \(a=\begin{pmatrix} 1\\ 0 \end{pmatrix}\) and
  \(b=\begin{pmatrix} 0\\ 1 \end{pmatrix}\) --\textgreater{} Ale
  Linearly Independent\\
\end{enumerate}

choose \(\lambda_1a +\lambda_2b= 0\)\\
 \[<=>\]

\[\begin{cases} \lambda_1= 0\\ \lambda_2 = 0 \end{cases}\]\\

The concept extends to \ldots{}. \(\forall\) number of vectors.\\

Set of vectors \(\underline{x_1},...\underline{x_i}\), for
\(\underline{x_i} \in\) \(R^{n}\)\\

is called \(\textbf{LINEARLY INDEPENDENT}\) if for \(\forall\)
\(\lambda_1,...,\lambda_p\) \(\in\) R,
\(\lambda^2_1+\lambda^2_2+\lambda^2_3+...+\lambda^2_p > 0\)\\

\[\sum\limits_{k=1}^p \lambda_k \underline{x_k} \neq \underline{0}\]

Otherwise , they are linearly dependent.\\

\paragraph{\texorpdfstring{EXAMPLE\\
}{EXAMPLE }}\label{example}

The Vectors:\\
 \(\begin{pmatrix} 0\\ 3\\2 \end{pmatrix}\),
\(\begin{pmatrix} 1\\ -1\\-3 \end{pmatrix}\) ,
\(\begin{pmatrix} 2\\ 2\\-1 \end{pmatrix}\) and
\(\begin{pmatrix} 1\\ 13\\11 \end{pmatrix}\) are Linearly Dependent.

\paragraph{\texorpdfstring{RANK OF MATRIX\\
}{RANK OF MATRIX }}\label{rank-of-matrix}

\begin{tcolorbox}[colback=green!5,colframe=red!40!black,title=Definition]
Rank of a matrix is the largest number of Linear Independent or Rows. 



\end{tcolorbox}

\begin{tcolorbox}[colback=green!5,colframe=red!40!black,title=Definition]
A square matrix $A_{nxn}$ is called $\textbf{Full-Rank}$ if rank $(A)=n$.



\end{tcolorbox}

Full rank square matrices have inverses \ldots{}.. is also true.

\paragraph{\texorpdfstring{EXAMPLE(REVISIT)\\
}{EXAMPLE(REVISIT) }}\label{examplerevisit}

\[A=\begin{pmatrix} 1 & 3 \\ 2 & 6 \end{pmatrix}\]

-\textgreater{}
\(\text{\textit{Note that column linearly dependent.}}\)\\

\(-3*\begin{pmatrix} 1 \\ 2 \end{pmatrix} + 1*\begin{pmatrix} 3 \\ 6 \end{pmatrix}\)
= 0\\

\begin{tcolorbox}[colback=green!5,colframe=blue!40!black,title=Theorem]
If A is non-singular then Rank(AB)=Rank(B)=Rank(BA). 



\end{tcolorbox}

\section{\texorpdfstring{6. Determinant of a Matrix\\
}{6. Determinant of a Matrix }}\label{determinant-of-a-matrix}

-You may remember determinants from your linear algebra courses mean
applying \(\textbf{Crame's Rule }\) to solving
\(\textbf{systems of linear equations}\), also \(\textbf{Jacobians}\).

-Determinants are defined for \(\textbf{square matrices}\).

-Geometrically, the absolute value of determinant is the volume of the
parallelpiped spanned by its column vectors.

-The determinant of matrix \(A\) is denoted by \(|A|\).

-The determinant of a 2x2 is
\(\begin{pmatrix} a & b \\ c & d \end{pmatrix}\)=\(ad-bc\)

-Direct calculation of determinants for matrices of larger sizes is
combersome and usually some decomposition method is used. Calculating
3x3 is still ok:\\[2\baselineskip]
\[\begin{pmatrix} a_{11} & a_{12} & a_{13} \\ a_{21} & a_{22} & a_{23}\\a_{31} & a_{32} & a_{33} \end{pmatrix}=a_{11}a_{22}a_{33}+a_{21}a_{32}a_{13}+a_{12}a_{23}a_{31}-a_{13}a_{22}a_{31}-a_{12}a_{21}a_{33}-a_{11}a_{32}a_{23}\]\\

\paragraph{\texorpdfstring{PROPERTIES OF DETERMINANT\\
}{PROPERTIES OF DETERMINANT }}\label{properties-of-determinant}

\begin{enumerate}
\def\labelenumi{\alph{enumi})}
\item
  \(|A|=|A'|\)
\item
  \(|AB|=|A||B|\)
\item
  \(|A|=0\) \textless{}=\textgreater{} \(A\) is singular
\item
  If \(\exists A^{-1}\) then \(|A'|\)=\(\frac{1}{|A|}\)
\item
  \(|kA|= k^{n}|A|\), if \(A_{nxn}\)
\item
  Interchanging 2 rows (columns) changes the sign of determinant.
\item
  Multiplying of an entire row (column) by a constant multiplies the
  determimant by that constant.
\item
  Adding a scalar multiple of one row (column) to another row (column)
  does not change the value of the determinant.
\item
  If matrix is triangular , determinant is a product of diagonal
  elements.
\end{enumerate}

\paragraph{\texorpdfstring{EXAMPLE\\
}{EXAMPLE }}\label{example-1}

\begin{enumerate}
\def\labelenumi{\arabic{enumi}.}
\item
  A=\(\begin{pmatrix} 1 & 2 \\ 1 & d-1 \end{pmatrix}\), then \(|A|\) =
  \(1*(-1)-1*2=-1-2=-3\)\\
\item
  \(\begin{pmatrix} 1 & -1 \\ 1 & 2 \end{pmatrix}\) = \(1*2-1*(-1)=3\)
  --\textgreater{}(f)\\
\item
  \(|2*\begin{pmatrix} 1 & -1 \\ 1 & 2 \end{pmatrix}|\) =
  \(\begin{pmatrix} 2 & 4 \\ 2 & -2 \end{pmatrix}\) =
  \(-4-8=-12=2^2*(-3)\) --\textgreater{}(e)\\
\end{enumerate}

4.\(\begin{pmatrix} 5 & 2 \\ 5 & -1 \end{pmatrix}\) =
\(-5-10=-15=5*(-3)\) --\textgreater{}(g)\\[2\baselineskip] 5. Let us
multiply col 1 by -2 and add to col 2\\[2\baselineskip]
\(\begin{pmatrix} 1 & 0 \\ 1 & -3 \end{pmatrix}\) = \(-3\) same as
\textbar{}A\textbar{} --\textgreater{}(h)\\

\section{\texorpdfstring{7. PATRITIONED MATRICES\\
}{7. PATRITIONED MATRICES }}\label{patritioned-matrices}

We can partition matrices into \(\textbf{blocks}\).\\

\paragraph{\texorpdfstring{EXAMPLE\\
}{EXAMPLE }}\label{example-2}

\(A=\left[\begin{array}{c|c}\begin{matrix} 1 & 2 \\ 3 & 6 \end{matrix} & \begin{matrix} 0 & 4 \\ 8 & 0 \end{matrix} \\ \hline \begin{matrix} 1 & 2 \end{matrix} & \begin{matrix} 0 & 1\end{matrix} \end{array} \right]=\left[\begin{array}{c|c}A_{11} & A_{12} \\ \hline A_{21} & A_{22} \end{array} \right]\)

\(A_{11}=\begin{pmatrix} 1 & 2 \\ 3 & 6 \end{pmatrix}\),
\(A_{12}=\begin{pmatrix} 0 & 4 \\ 8 & 0 \end{pmatrix}\),
\(A_{21}= \begin{pmatrix} 1 & 2\end{pmatrix}\),
\(A_{22}= \begin{pmatrix} 0 & 1\end{pmatrix}\)\\

As long as dimensions conform, matrix addition , multiplication,
transposition work the same way. That is the power of PARTITIONING.\\

\paragraph{\texorpdfstring{EXAMPLE\\
}{EXAMPLE }}\label{example-3}

\(\underline{b}=\left[\begin{array}{c|c} 2 \\ 0 \\ \hline -1 \\ 2 \end{array}\right]\)
= \(\begin{pmatrix} b_1 \\ b_2\end{pmatrix}\)\\

Let's do multiplication with partitioned matrices:\\

\(A\underline{b}=\left[\begin{array}{c|c}A_{11} & A_{12} \\ \hline A_{21} & A_{22} \end{array} \right]\begin{pmatrix} b_1 \\ b_2\end{pmatrix}\)
=
\(\begin{pmatrix} A_{11}b_1 + A_{12}b_2\\ A_{21}b_1 + A_{22}b_2\end{pmatrix}\),
where:\\

\(A_{11}b_1=\begin{pmatrix} 2 \\ 6\end{pmatrix}\) ;
\(A_{12}b_2=\begin{pmatrix} 8 \\ -8\end{pmatrix}\) ; \(A_{21}b_1=(2)\);
\(A_{22}b_2=(2)\);\\
 So: \(AB = \begin{pmatrix} 10 \\ -2 \\ 4\end{pmatrix}\)\\
 Of course we get the same answer by performing multiplication on
original matrices. (DO IT!)\\

\paragraph{\texorpdfstring{EXAMPLE\\
}{EXAMPLE }}\label{example-4}

If R\&S are non-singular show that :
\(\begin{pmatrix} R & 0 \\ L & S\end{pmatrix}^{-1}\) =
\(\begin{pmatrix} R^{-1} & 0 \\ -S^{-1}LR^{-1} & S^{-1}\end{pmatrix}\)\\

PROOF ---\textgreater{} Provide.

\paragraph{EXAMPLE~}\label{example-5}

If A is a block diagonal matrix:\\

\(A\)=\(\begin{bmatrix} A_{11} & \cdots & 0 \\ \vdots & \ddots & \vdots \\ 0 & \cdots & A_{nn} \end{bmatrix}\)\\[2\baselineskip]Then\\
 \(det(A)=det(A_{11})*det(A_{11})*det(A_{22})...det(A_{nn})\)

Let:\\

\(A =\begin{pmatrix} 2 & 0 & 0\\ 0 & 4 & 5 \\ 0 & 6 & 7 \end{pmatrix}\)\\

Find det(A) ---\textgreater{} Provide solution.

\section{\texorpdfstring{7. SOME SPECIAL MATRICES\\
}{7. SOME SPECIAL MATRICES }}\label{some-special-matrices}

\begin{itemize}
\item
  SYMMETRIC MATRICES\\

  Square matrix A, \(A^{T}=A\)\\
   Variance - Covariance matrices are symmetric.\\
\item
  ORTHOGONAL MATRICES (RIGID TRANSFORMATION)\\

  Two vectors \(\underline{x}\) \& \(\underline{y}\) are called
  \(\textbf{ORTHOGONAL}\) if \(\underline{x^{T}}\) \(\underline{y}\) = 0
  (Dot Product)\\

  Two vectors \(\underline{x}\) \& \(\underline{y}\) are called
  \(\textbf{ORTHOGONAL}\) if:\\

  \begin{enumerate}
  \def\labelenumi{\arabic{enumi})}
  \item
    They are orthogonal and
  \item
    \(\underline{x^{T}}\) \(\underline{x}\) = 1 ; \(\underline{y^{T}}\)
    \(\underline{y}\) = 1 (Unit Length)\\
  \end{enumerate}
\item
  A square matrix Q is called \(\textbf{ORTHOGONAL}\) if its columns are
  ORTHOGONAL.\\
\end{itemize}

\paragraph{\texorpdfstring{EXAMPLE\\
}{EXAMPLE }}\label{example-6}

Is
\(A =\begin{pmatrix} \frac{1}{\sqrt{2}} & -\frac{1}{\sqrt{2}} \\ -\frac{1}{\sqrt{2}} & \frac{1}{\sqrt{2}}\end{pmatrix}\)
orthonormal? Show it.\\

\begin{itemize}
\item
  Properties of ORTHOGONAL Matrices.\\

  \begin{enumerate}
  \def\labelenumi{\roman{enumi})}
  \tightlist
  \item
    \(Q^{T}Q=I\)\\
  \item
    \(Q^{-1}=Q^{T}\)\\
  \item
    \textbar{}delta\textbar{} = 1 ---\textgreater{} det is +/- 1\\
  \end{enumerate}
\item
  Idempotent Matrices.\\

  A is idempotent \textless{}--\textgreater{} \(A^{2}=A\)

  If A is idempotent than \((I - H)\) is too.
\end{itemize}

\paragraph{\texorpdfstring{EXAMPLE\\
}{EXAMPLE }}\label{example-7}

The hat matrix from OLS

\(H=X(X'X)^{-1}X'\) is idempotent. Show it!

\subsection{\texorpdfstring{PERMUTATION MATRICES\\
}{PERMUTATION MATRICES }}\label{permutation-matrices}

\(\textbf{P}\) is called a \(\textbf{Permutation Matrix}\) if it can be
obtained from identity matrix by permuting columns.\\

-Properties:

\begin{enumerate}
\def\labelenumi{\alph{enumi})}
\tightlist
\item
  Permutation Matrices are \(\textbf{orthogonal}\). Their determinant is
  +/- 1.\\
   Also if \(|A| = a\) and \(\textbf{P}\) is a Permutation Matrix , then
  \(|AP|\) = \(|PA|\) = +/- a\\
\end{enumerate}

\subsection{\texorpdfstring{DIAGONAL MATRICES\\
}{DIAGONAL MATRICES }}\label{diagonal-matrices}

If only non zero elements are on diagonal:\\

\(I_{nxn}\)=\(\begin{bmatrix} a_{11} & \cdots & 0 \\ \vdots & \ddots & \vdots \\ 0 & \cdots & a_{nn} \end{bmatrix}\)\\
 Properties:

\begin{enumerate}
\def\labelenumi{\alph{enumi})}
\tightlist
\item
  Determinant = \(a_{11}*a_{22}*...*a_{nn}\)\\
\item
  \(\prod_{i=1}^{n}a_{ii} \neq 0\) \textless{}=\textgreater{} Full
  Rank\\
\end{enumerate}

\begin{equation}
  \left(
    \begin{array}{*5{c}}
     x & x & x & x & x \\
     0 & x & x & x & x \\
     0 &  0 & x & x & x \\
     0 &  0 &  0 & x & x \\
     0 &  0 &  0 &  0 & x \\
  \end{array}\right)
\end{equation}

---\textgreater{} UPPER TRIANGULAR

\begin{equation}
  \left(
    \begin{array}{*5{c}}
     0 & 0 & 0 & 0 & 0 \\
     x & 0 & 0 & 0 & 0 \\
     x &  x & 0 & 0 & 0 \\
     x &  x &  x & 0 & 0 \\
     x &  x &  x &  x & 0 \\
  \end{array}\right)
\end{equation}

---\textgreater{} LOWER TRIANGULAR\\

Properties ( the same as for diagonal matrices):

\begin{enumerate}
\def\labelenumi{\alph{enumi})}
\tightlist
\item
  Determinant = \(a_{11}*a_{22}*...*a_{nn}\)\\
\item
  \(\prod_{i=1}^{n}a_{ii} \neq 0\) \textless{}=\textgreater{} Full
  Rank\\
\end{enumerate}

\section{\texorpdfstring{8. COLUMN REDUCTION\\
}{8. COLUMN REDUCTION }}\label{column-reduction}

Column reduction of matrices is used for various purposes, including:\\

-Determinig if matrix is full rank\\[2\baselineskip] -Calculating
inverse.\\[2\baselineskip] -Calculating the
determinant.\\[2\baselineskip] -Solving system of linear equations.\\

Column reduction involves 3 types of so called elementry column
operations.\\

Each elementry column operation is equivalent to multiplying the matrix
from the right by a special type of matrix.\\

\end{document}
